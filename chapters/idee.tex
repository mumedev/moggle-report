% 0.5 blz
\chapter{Idee}\label{chapter:idee}

De applicatie die we zullen ontwerpen en implementeren is een Quantified Self applicatie getiteld 'Moggle', een woordspeling op 'mood' en 'Toggl'. Het basisidee was om de werkuren die gepresteerd werden door de gebruiker manueel te laten ingeven, of indien mogelijk automatisch laten importeren via Toggle, en vervolgens zijn prestaties te laten evalueren volgens drie indicatoren: slecht, gemiddeld en goed. Ten slotte geeft de gebruiker zijn gemoedstoestand in. Het resultaat is dus een datamodel waarin de werkkwantiteit en -kwaliteit aan de gemoedstoestand gekoppeld worden.

Uitbreidingen hierop zouden kunnen zijn om ook het tijdstipaspect in rekening te brengen (e.g. dag van de week), werklocatie, achtergrondgeluiden tijdens het werk, de specifieke activiteit (studeren, programmeren, trainen, ...) en dergelijke meer. Het automatisch 'tracken' van zulke parameters is waarschijnlijk niet eenvoudig, aangezien het toestel niet noodzakelijk aanstaat tijdens het presteren van de uren zelf.