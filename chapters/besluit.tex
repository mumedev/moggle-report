% 1 Bladzijde
\chapter{Besluit}\label{chapter:besluit}

Het doel van de cursus 'Multimedia: modelleren en programmeren' bestaat erin een mobiele applicatie te ontwerpen en implementeren in verschillende technologie\"en (iOS, Android, HTML5, ...) om uiteindelijk inzicht te verwerven in de sterktes en zwaktes van deze technologie\"en. In dit verslag werden achtereenvolgens een scenario, het storyboard, de applicatiearchitectuur het het softwareontwerp voor zowel de Android- als de iOS-versie van de Moggle applicatie besproken. Ten slotte werden properties van beide technologie\"en overlopen en vergeleken.

\section{Vergelijking technologie\"en}\label{section:vergelijking_technologieen}

Hoewel de ondersteuning voor zowel Android als iOS vrij gelijklopend is, is het programmeren in iOS minder toegankelijk. Zowel op vlak van kosten, op vlak van distributie en op vlak van platformen scoort iOS minder goed. Men beoogt meer uniformiteit in applicaties en veiligere software in Apple App Store.

Het model-view-controller patroon wordt ondersteund in beide technologie\"en. Kennelijk is er kritiek op de implementatie door Android, maar al bij al volgen zowel iOS als Android een vrij gelijkaardige strategie. Geen van beide ondersteunt pure loose coupling. Een voordeel van iOS is het Core Data framework, dat een goede architectuur vormt voor het model.

Het maken van layouts voor views in Android is iets transparanter dan voor iOS - al vindt ik persoonlijk het Wysiwyg drag-and-drop systeem met storyboards beter dan de uitwerking in Android.

Het aanleren van Android was voor mij redelijk intu\"itief, aangezien ik reeds vertrouwd was met Java en XML. Dit hielp uiteraard ook bij het verwerken van tutorials en documentatie. iOS vond ik minder transparant. Het feit dat ik pas vrij laat met de meer gestructureerde aanpak van de Stanford Slides\footnote{\url{http://www.stanford.edu/class/cs193p/cgi-bin/drupal/downloads-2011-fall}} ben begonnen, heeft vooral voor verwarring gezorgd bij de aanvang van de cursus. Mede door het feit dat er veel verouderde informatie online staat, waarin nog geen gebruik gemaakt wordt van storyboards.

Op basis van de aangehaalde argumenten zou kunnen geconcludeerd worden dat op vlak van toegankelijkheid en ondersteuning Android beter scoort dan iOS. Op vlak van architectuur lijkt iOS de bovenhand te hebben over Android.

% the question is porbably :
% is it worth developing in both?
% rather than which is better?
% market share

\section{Cursus}

De cursus is zeker relevant voor een opleiding in softwareontwikkeling en multimedia. De aangeboden topics zijn interessant en ook leuk om mee te werken - ondanks de vele frustraties die gepaard gaan met software development en het leren van een nieuwe programmeertaal.

De cursus is nog relatief nieuw en staat nog niet helemaal op punt, heb ik de indruk. Naar mijn mening mist de cursus wat structuur. Enkele bedenkingen:

\begin{itemize}
	\item doelstellingen mogen nog concreter geformuleerd worden bij aanvang van de cursus: zowel globale doelstellingen als concrete doelstellingen betreffende implementatie;
	\item een beter overzicht geven van websites, video's, boeken en cursussen die goed bleken te zijn;
	\item best practices en richtlijnen voor het schrijven van blogposts: sommige posts missen structuur en lijken het globale verhaal van de blog te missen;
	\item anders kaderen van presentaties door externen in het geheel van de lessen: omwille van het estaffettesysteem was de timing van sommige presentaties misschien wat ongelukkig. Er leek ook niet echt een logica te zitten in de opeenvolging van de presentaties; % meer state-of-the-art, case studies, timing
\end{itemize}






