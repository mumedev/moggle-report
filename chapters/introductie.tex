% introductie 0.5 blz
\chapter{Introductie}\label{chapter:introductie}

Dit is het eindverslag voor de cursus 'Multimedia: modelleren en pogrammeren'. Het doel van deze cursus is om een mobiele applicatie te ontwerpen en implementeren in verschillende technologi\"en (iOS, Android, HTML5, ...) om uiteindelijk inzicht te verwerven in de sterktes en zwaktes van deze technologi\"en.

De applicatie is een Quantified Self App. Dit soort toepassing vraagt de gebruiker om invoer, en/of verzamelt deze automatisch afhankelijk van het type data, gedurende een zekere periode. Deze data kan dan gevisualiseerd worden om mogelijk gedragspatronen te onthullen\cite{govaerts2012}.

De technologie\"en die worden besproken in dit verslag zijn iOS en Android. Alle code voor dit project is open source en kan gevonden en gedownloaded worden van github\footnote{\url{https://github.com/JorisSchelfaut/mumedev}}. Naast dit verslag kan informatie over het project gevonden worden op de wordpress blog\footnote{\url{http://mumedev.wordpress.com/}} en youtube vlog\footnote{\url{http://www.youtube.com/user/mumedev}}.

We bespreken eerst het idee achter de applicatie. Vervolgens kijken we naar een scenario en daarna naar het storyboard van de applicatie. Dan wordt de architectuur en het software ontwerp van de applicatie voor de verschillende technologie\"en bekeken. Hierna maken we een vergelijking tussen de verschillende technologie\"en. We eindigen met een besluit.