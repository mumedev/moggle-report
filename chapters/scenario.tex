% scenario 0.5 blz
\chapter{Scenario}\label{chapter:scenario}

We beschrijven een kort scenario van een persoon die de applicatie gebruikt om na te gaan of er een correlatie zou kunnen bestaan tussen zijn geleverde werk en zijn gemoedstoestand.\\

%{\addtolength{\leftskip}{5mm} ...text... }

Het is maandag. Een student heeft net een vijftal uren aan een verslag gewerkt. Hij houdt het voor bekeken voor die dag en vult zijn gepresteerde uren in op de Moggle applicatie, en ook de rating van zijn werk en zijn gemoedstoestand. Hij is tevreden over zijn werk en geeft zijn gemoedstoestand een rating van 9 op 10.

Woensdag, donderdag en zaterdag speelt zich eenzelfde scenario af met respectievelijk 2, 5 en 7 uren gewerkt, 'slecht', 'goed' en 'matig' als werkkwaliteit en 3, 7 en 7 als ratings voor zijn gemoedstoestand.

Wanneer hij zijn resultaten bekijkt, merkt hij dat slecht en weinig werk leveren doorgaans correleert met een slechte gemoedstoestand, terwijl goed werk leveren vooral overeenstemt met een goede gemoedstoestand. Veel en goed werk leveren lijkt echter niet altijd te correleren met een goede gemoedstoestand.
